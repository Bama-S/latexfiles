\documentclass[a4paper, 12 pt]{article}

\usepackage{amsmath}    % need for subequations
\usepackage{graphicx}   % need for figures
\usepackage{verbatim}   % useful for program listings
\usepackage{color}      % use if color is used in text
  % use for side-by-side figures
\usepackage{hyperref}   % use for hypertext links, including those to external documents and URLs

% don't need the following. simply use defaults
\setlength{\baselineskip}{16.0pt}    % 16 pt usual spacing between lines

\setlength{\parskip}{3pt plus 2pt}
\setlength{\parindent}{20pt}
\setlength{\oddsidemargin}{0.5cm}
\setlength{\evensidemargin}{0.5cm}
\setlength{\marginparsep}{0.75cm}
\setlength{\marginparwidth}{2.5cm}
\setlength{\marginparpush}{1.0cm}
\setlength{\textwidth}{150mm}
\title{Knowledge and Women}
\begin{document}
\maketitle
\newpage
%\tableofcontents

Since ancient times, women have been held at an esteemed position in terms of knowledge. This paper tries to address the role of women who have raised the scientific levels to great heights both from the Western and Indian Perspective. From the western viewpoint,  there has been evidences of Greeks and Romans worshipping Goddess Athena and Minerva, respectively. On a similar note, Indians have regarded Goddess Saraswati as the mother of knowledge and wisdom. The intellectual calibre of Indian women has been explicitly stated in the epics and puranas. For example, there is a mention in Mahabharata indicating the managerial capabilities of Draupadi. These characters, albeit mythological portray the essence of the limitless capabilities of women.

Starting from the historic times, the western world has seen women professionals. Some of them are Merit-Ptah the first known physician from Ancient Egypt, Agamede the first female healer from Greece, Hypatia first known woman mathematician, Charlotta Frolich first female historian,  Marie Curie first woman to win a nobel prize, Laura Bassi first woman to earn a university chair (18th century),  Ada Lovelace first woman programmer, Grace Hopper woman who invented first compiler and Mary Anning Paleontologist.

A few remarkable Indians are Avvaiyar, who was one of the greatest poets of all times,  Gargi and Maithreyi, who are mentioned in the Upanishad texts,  the saint poet Mirabhai, Leelavathi, the mathematician and astrologer, Razia Sultana a monarch, the south asian physicians Kadambini Ganguly and Anandhi Gopal Joshi, the meteorologist Anna Manni, Sarla Takral who was the first woman to fly an aircraft and in the present scenario Dr. Shantha, who is heading the Adayar Cancer Institute. 

This paper explores the effort and dedication put forth by the aforementioned personalities with a hope that it will create an inspiration to the present generation female individuals.





%\bibliographystyle{apalike}
%{\small
%\bibliography{collection}}
%\vfill

\end{document}
